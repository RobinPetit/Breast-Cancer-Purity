\documentclass[letterpaper]{article}

\usepackage[utf8]{inputenc}
\usepackage[sort, colon]{natbib}
\usepackage{alifexi}
\usepackage[bottom]{footmisc}
\usepackage[colorlinks=true,citecolor=green,linkcolor=blue]{hyperref}
\usepackage[font=small,labelfont=bf]{caption}
\usepackage{subcaption}
\usepackage{tabulary}
\usepackage{flushend}
\usepackage{dblfloatfix}
\usepackage[export]{adjustbox}

\usepackage{xcolor}  % To be removed in the end

\usepackage{amsmath}
\usepackage{commath}

\urlstyle{same}

\newcommand{\TODO}[1]{\textcolor{red}{TODO: {#1}}}

\title{BINF-F401 --- \textit{Analysis of functional and comparative genomic data}\\
Study of tumorous purity in breast cancer}
\author{Robin Petit$^\dagger$\\
\mbox{}\\
$^\dagger$robpetit@ulb.ac.be}
\date{June, 14 2018}

\begin{document}
\maketitle

\begin{abstract}
	This is not an abstract
\end{abstract}

\section{Introduction}
Table~\ref{tab:abbrev} contains all the abbreviations used in this paper. Please refer to this table in case of incomprehension.

\begin{table}[!h]
\centering
\begin{tabular}{|c|c|}
\hline
Abbreviation & Meaning \\ \hline \hline
BRCA & Breast Cancer \\ \hline
CDF & Cumulated Density Function \\ \hline
ER & Estrogen Receptor \\ \hline
GSE & Gene Set Enrichment \\ \hline
IHC & Immunohistochemistry \\ \hline
LCIS & Lobular Breast Carcinoma \textit{in situ} \\ \hline
NMF & Non-Negative Matrix Factorization \\ \hline
PDF & Probability Density Function \\ \hline
PR & Progesterone Receptor \\ \hline
TCGA & The Cancer Genome Atlas \\ \hline
\end{tabular}
\caption{Meaning of common abbreviations used in this paper.\label{tab:abbrev}}
\end{table}

The purity of a tumor is the proportion of cancerous cells that is present in the tumor. It is pathologists' responsibility to determine this purity
by analysing histopathological slides of the tumor and counting all the different types of cells. Yet, a pathologist is unable to analyse the whole
tumor which would take an enormous amount of time. Therefore, depending on the objective of the slide analysis, several computer-aided methods have
been developed to either segment the slides~\citep{sirinukunwattana2016locality,xing2016robust,komura2017machine}, or analyse the tumor genetically
to find the somatic DNA alterations~\citep{carter2012absolute} or even use whole-genome and whole-exome sequencing~\citep{oesper2014quantifying}.

The Cancer Genome Atlas (TCGA)~\citep{weinstein2013cancer} has become an important database for cancer-related data. 250 samples have been taken from there in order to
analyze tumorous purity (both ABSOLUTE~\citep{carter2012absolute} and IHC) through mutations in genes (see Table~\ref{tab:genes}), NMF of mRNA-seq
and GSE~\citep{subramanian2005gene}.

\begin{table*}[!t]
\centering
\vspace{-.5cm}
\begin{tabular}{c|c|c}
HGNC Approved Symbol & Entrez ID & Related Cancer Type \\ \hline \hline
CDH1 & 999 & LCIS~\citep{berx1998mutations} \\ \hline
GATA3 & 2625 & ER-$\alpha$ related~\citep{ciocca2009significance} \\ \hline
MAP3K1 & 4214 & Invasive BRCA~\citep{easton2007genome} \\ \hline
PIK3CA & 5290 & Invasive BRCA and ER/PR-related~\citep{saal2005pik3ca} \\ \hline
TP53 & 7157 & BRCA~\citep{gasco2002p53}
\end{tabular}
\caption{Genes considered in this study with both their HGNC symbol which (used further down) and their Entrez ID.\label{tab:genes}}
\end{table*}

\TODO{extend introduction...}

\newpage
\section{Results}
\subsection{Comparison between IHC and ABSOLUTE purities}
When comparing IHC purity versus ABSOLUTE purity, we observed that these measures only slightly correlate: Spearman's $r_S = 0.355$ and Pearson's $r_P = 0.36$.
Yet, they do correlate significantly with $p < 10^{-8}$.

\begin{figure}[!h]
%\vspace{-.5cm}
\hspace{-.6cm}
\includegraphics[width=.5\textwidth,trim={0 0 0 1.5cm},clip]{figs/abs_vs_ihc.eps}
\vspace{-1cm}
\caption{\textit{(Up)} Couples (IHC, ABSOLUTE) purities. The dot-dashed line is the theoretical $\text{IHC} = \text{ABSOLUTE}$ curve. It is known that pathologists
are biased on their work~\citep{fandel2008we}, in this case, IHC purities seem to overestimate the purity of tumors.
\textit{(Down)} Frequency of difference between IHC purity and ABSOLUTE purity.\label{fig:abs vs ihc}}
\end{figure}

More specifically, IHC measures tend to be greater than ABSOLUTE evaluations (Figure~\ref{fig:abs vs ihc}) with a mean of $\text{IHC} - \text{ABSOLUTE}$ of $0.202$
and a standard deviation of $0.175$. This indicates that either pathologists overestimate homogeneity in tumors or ABSOLUTE underestimates it. It must be kept in
mind that ABSOLUTE has been known to underestimate purity in certain cases~\citep{oesper2014quantifying}. Still, ABSOLUTE estimations are known to be highly
accurate~\citep{carter2012absolute}.

\subsection{Influence of mutations on ABSOLUTE purity}
The distribution of the number of mutations of each gene from Table~\ref{tab:genes} is shown on Figure~\ref{fig:mutations distribution}. We can observe that for each
gene, more than $50\%$ of samples are not mutated, and that CDH1, GATA3, MAP3K1 are highly non-mutated (more than $80\%$ of samples).

Though, when summing the mutations of every gene, the mode becomes 1 mutation with only around $25\%$ of samples having no mutations at all, therefore around $75\%$
of samples have at least one of these five genes which is mutated.

\begin{figure}[!h]
\hspace{-.5cm}
{\includegraphics[width=.55\textwidth,trim={0 0 0 2cm},clip]{figs/mutations_distribution.eps}
\vspace{-1cm}
\caption{Distribution of the number of mutations of each gene from Table~\ref{tab:genes} and of the sum on all genes.\label{fig:mutations distribution}}}
\end{figure}

\begin{figure*}[!t]
\begin{subfigure}{.49\textwidth}
\vspace{-.75cm}
\hspace{-.5cm}
\includegraphics[width=1.1\textwidth]{figs/nb_muts_vs_abs.eps}
\vspace{-1cm}
\subcaption{Relation between ABSOLUTE purity and number of mutations for each gene and for the cumulation of every genes. Significance of the dependence of these two quantities
is indicated next to the name of the concerned gene.\label{fig:nb mutations vs abs}}
\end{subfigure}
\begin{subfigure}{.49\textwidth}
\vspace{-.75cm}
\includegraphics[width=1.1\textwidth]{figs/nb_muts_vs_abs_binarized.eps}
\vspace{-1cm}
\subcaption{Relation between ABSOLUTE purity and presence of mutations for each gene and for the cumulation of every genes. Significance of the dependence of these two quantities
is indicated next to the name of the concerned gene.\label{fig:nb mutations vs abs binarized}}
\end{subfigure}
\begin{subfigure}{.49\textwidth}
\vspace{-.75cm}
\hspace{-.5cm}
\includegraphics[width=1.1\textwidth]{figs/nb_muts_vs_ihc.eps}
\vspace{-1cm}
\subcaption{Adaptation of Figure~\ref{fig:nb mutations vs abs} for IHC purity.\label{fig:nb mutations vs ihc}}
\end{subfigure}
\begin{subfigure}{.49\textwidth}
\vspace{-.75cm}
\includegraphics[width=1.1\textwidth]{figs/nb_muts_vs_ihc_binarized.eps}
\vspace{-1cm}
\subcaption{Adaptation of Figure~\ref{fig:nb mutations vs abs binarized} for IHC purity.\label{fig:nb mutations vs ihc binarized}}
\end{subfigure}
\caption{Relation between mutations and ABSOLUTE/IHC purity.}
\end{figure*}

Moreover, the only pairs of genes showing significant correlation ($p < 0.05$) in the number of mutations are (CDH1, GATA3), (CDH1, TP53), (MAP3K1, TP53)
and (PIK3CA, TP53). All the other pairs cannot be considered correlated (see Table~\ref{tab:correlations nb mutations}). Also, these correlations are negative and light
($-0.21 \leq r \leq -0.127$) meaning that a bigger number of mutations in TP53 is correlated with a lower number of mutations of CDH1, MAP3K1 and PIK3CA, and a
bigger number of mutations of GATA3 is correlated with a lower number of mutations of CDH1.

When aggregating all the mutations together, the correlation coefficient of these four gene pairs stays roughly constant except for (PIK3CA, TP53) for which the
correlation increased from $-0.127$ to $-0.161$. Also a new pair of genes has become significantly correlated: (CDH1, PIK3CA); and except that one pair,
no pair has changed significance threshold. We also noticed that this last pair has positive correlation.

\begin{table}[!h]
\begin{subtable}{.5\textwidth}
\begin{tabular}{r|c|c|c|c}
       & GATA3        & MAP3K1   & PIK3CA       & TP53           \\ \hline
CDH1   & $-0.129^{*}$ & $-0.034$ & $+0.114$     & $-0.210^{***}$ \\ \hline
GATA3  &              & $+0.011$ & $-0.114$     & $-0.078$       \\ \hline
MAP3K1 &              &          & $+0.058$     & $-0.170^{**}$  \\ \hline
PIK3CA &              &          &              & $-0.127^{*}$
\end{tabular}
\subcaption{Pearson's $r$ coefficient of the number of mutations for each pair of genes.\label{tab:correlations nb mutations}}
\end{subtable}
\begin{subtable}{.5\textwidth}
\begin{tabular}{r|c|c|c|c}
       & GATA3        & MAP3K1   & PIK3CA       & TP53           \\ \hline
CDH1   & $-0.131^{*}$ & $-0.037$ & $+0.125^{*}$ & $-0.214^{***}$ \\ \hline
GATA3  &              & $+0.008$ & $-0.107$     & $-0.075$       \\ \hline
MAP3K1 &              &          & $+0.073$     & $-0.175^{**}$  \\ \hline
PIK3CA &              &          &              & $-0.161^{*}$
\end{tabular}
\subcaption{Pearson's $r$ coefficient of the presence of mutations for each pair of genes.\label{tab:correlations binarized}}
\end{subtable}
\caption{Correlation between the number of mutations or the presence of mutations for each pair of genes from Table~\ref{tab:genes}. \\
{\scriptsize Legend: $^{*}$: $p < 0.05$, $^{**}$: $p < 0.01$, $^{***}$: $p < 0.001$}.\label{tab:correlations}}
\end{table}

No dependence between the number of mutations of any gene and the ABSOLUTE purity (nor from the sum of all mutations and the ABSOLUTE purity) was detectable
(Figure~\ref{fig:nb mutations vs abs}): these quantities are independent to one another. Even when aggregating all the mutations such that each gene is either
mutated or non-mutated, no dependence is noticeable (Figure~\ref{fig:nb mutations vs abs binarized}).

The same result stands for the independence of the mutations (or number of mutations) and the IHC purity (see Figures~\ref{fig:nb mutations vs ihc}
and~\ref{fig:nb mutations vs ihc binarized}).

\subsection{Clustering}

\TODO{Detail Figures~\ref{fig:clustering}~and~\ref{fig:boxplot purity}}

\begin{figure}
\hspace{-.5cm}
\begin{subfigure}{.5\textwidth}
\vspace{-.75cm}
\includegraphics[width=1.1\textwidth]{figs/silhouette.eps}
\subcaption{Silhouette analysis for NMF clustering with $7$ clusters.\label{fig:silhouettes}}
\end{subfigure}
\begin{subfigure}{.5\textwidth}
\includegraphics[width=1.1\textwidth]{figs/clustered_silhouette_boxplot.eps}
\subcaption{Distribution of silhouette width for each cluster.\label{fig:boxplot silhouette}}
\end{subfigure}
\caption{Clustering of the samples and silhouette distribution.\label{fig:clustering}}
\end{figure}

\begin{figure}
\vspace{-.25cm}
\hspace{-.5cm}
\includegraphics[width=.55\textwidth]{figs/clustered_purity_boxplot.eps}
\caption{Distribution of ABSOLUTE purity for each cluster.\label{fig:boxplot purity}}
\end{figure}

\section{Materials and Methods}
All of the (Python3) source code used in this study as well as this very document are available at the following web page: \url{https://github.com/RobinPetit/Breast-Cancer-Purity}.

The data used in this study are those provided for the project, comprising 250 samples with both IHC and ABSOLUTE purities and mutations of each gene of Table~\ref{tab:genes}
as well as those required to be downloaded.

\subsection{Significance of Pearson's $r$ correlation coefficient}
Significance indices ($p$-values) for Pearson's correlation coefficients are computed by a $t$-score: $t = r_P\sqrt{(N-2)/(1-{r_P}^2)}$ with $r_P$ being Pearson's $r$ and $N$
the sample size since this statistic follows a Student's t with $N-2$ degrees of freedom~\citep{lee1988thirteen} (with bivariate normal distribution assumption which
can be discarded provided the sample size is sufficiently big).

Figure~\ref{fig:r to p} shows the associated $p$-value of every correlation coefficient (in absolute value) as well as significance threshold $p = 0.05$, $p = 0.01$, and $p = 0.001$
and their associated quantiles. The lower right corner of the figure is a bit messy due to numerical stability issues. Nonetheless, the $p$-values keep decreasing for higher
values of $\abs r$.

\begin{figure}[!h]
\includegraphics[width=.5\textwidth]{figs/r_to_p.eps}
\caption{Associated $p$-value for $\abs r$.\label{fig:r to p}}
\end{figure}

For Figure~\ref{fig:abs vs ihc}, $N=248$ because samples TCGA-C8-A130-01 and TCGA-C8-A133-01 did not have provided IHC purity, so they were removed from the samples
for this correlation analysis.

\subsection{Independence tests}
Independence tests between two variables were performed using a $\chi^2$-test. For Figures~\ref{fig:nb mutations vs abs}~and~\ref{fig:nb mutations vs ihc}, the purity is a
continuous variable, therefore it has been discretized into 20 subintervals of equal size in order to make a proper contingency table for the $\chi^2$.

\section{Conclusion}
Blablabla...

\newpage
\footnotesize
\bibliographystyle{apalike}
\bibliography{report}{}
\end{document}
